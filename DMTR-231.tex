\documentclass[DM,lsstdraft,STR,toc]{lsstdoc}
\usepackage{geometry}
\usepackage{longtable,booktabs}
\usepackage{enumitem}
\usepackage{arydshln}
\usepackage{attachfile}
\usepackage{array}

\newcolumntype{L}[1]{>{\raggedright\let\newline\\\arraybackslash\hspace{0pt}}p{#1}}

\input meta.tex

\newcommand{\attachmentsUrl}{https://github.com/\gitorg/\lsstDocType-\lsstDocNum/blob/\gitref/attachments}
\providecommand{\tightlist}{
  \setlength{\itemsep}{0pt}\setlength{\parskip}{0pt}}

\setcounter{tocdepth}{4}

\begin{document}

\def\milestoneName{DM Acceptance Testing, Operations Rehearsal \#2}
\def\milestoneId{LVV-P72}
\def\product{Acceptance}

\setDocCompact{true}

\title{LVV-P72: DM Acceptance Testing, Operations Rehearsal \#2 Test Plan and Report}
\setDocRef{\lsstDocType-\lsstDocNum}
\date{\vcsdate}
\author{ Jeffrey Carlin }

% Most recent last
\setDocChangeRecord{
\addtohist{}{2020-08-19}{First draft}{Robert Gruendl}
\addtohist{1.0}{2020-09-04}{Test Plan LVV-P76 approved. \jira{DM-16196}.}{Robert Gruendl}
}

\setDocCurator{Robert Gruendl}
\setDocUpstreamLocation{\url{https://github.com/lsst-dm/\lsstDocType-\lsstDocNum}}
\setDocUpstreamVersion{\vcsrevision}



\setDocAbstract{
This is the test plan and report for LVV-P72 (DM Acceptance Testing, Operations Rehearsal \#2),
an LSST milestone pertaining to the Data Management Subsystem.
}


\maketitle

\section{Introduction}
\label{sect:intro}


\subsection{Objectives}
\label{sect:objectives}

 This Acceptance Test campaign aims to verify a small number of
\href{https://lse-61.lsst.io/}{DMSR} (\citeds{LSE-61}) requirements related to
the LSST Science Pipelines. It will be executed in conjunction with
Operations Rehearsal \#2. This Test Plan aims to demonstrate that the
included requirements have been met by the activities carried out during
the Operations Rehearsal, and to thus fully verify their completion and
readiness for LSST Operations.



\subsection{System Overview}
\label{sect:systemoverview}

 The tests to be executed are intended to verify that the DM system
satisfies a subset of the requirements outlined in the Data Management
System Requirements (DMSR; \href{https://lse-61.lsst.io/}{LSE-61} ).
This subset of requirements is related to pipeline algorithms, network
and observing facility infrastructure, and some basic camera and data
backbone functionalities. Additional DMSR requirements will be verified
in later Acceptance Test Campaigns.\\[2\baselineskip]The tests will be
performed using\ldots{}\\[2\baselineskip]Planning for the Operations
Rehearsal is being tracked at~
\href{https://confluence.lsstcorp.org/display/DM/Operations+Rehearsal+\%232}{this
Confluence page} .\\[2\baselineskip]\textbf{Applicable
Documents:}\\[2\baselineskip]\citeds{LSE-61} Data Management System
Requirements\\
\citeds{LDM-503} Data Management Test Plan\\
\citeds{LDM-639} LSST Data Management Acceptance Test Specification (issue 2.1)\\
\citeds{LDM-643} Proposed DM Ops Rehearsals (Chapter 3 in particular)\\
\citeds{LDM-732} Rubin Observatory Network Verification Baseline


\subsection{Document Overview}
\label{sect:docoverview}

This document was generated from Jira, obtaining the relevant information from the
\href{https://jira.lsstcorp.org/secure/Tests.jspa\#/testPlan/LVV-P72}{LVV-P72}
~Jira Test Plan and related Test Cycles (
\href{https://jira.lsstcorp.org/secure/Tests.jspa\#/testCycle/LVV-C154}{LVV-C154}
).

Section \ref{sect:intro} provides an overview of the test campaign, the system under test (\product{}),
the applicable documentation, and explains how this document is organized.
Section \ref{sect:testplan} provides additional information about the test plan, like for example the configuration
used for this test or related documentation.
Section \ref{sect:personnel} describes the necessary roles and lists the individuals assigned to them.

Section \ref{sect:overview} provides a summary of the test results, including an overview in Table \ref{table:summary},
an overall assessment statement and suggestions for possible improvements.
Section \ref{sect:detailedtestresults} provides detailed results for each step in each test case.

The current status of test plan \href{https://jira.lsstcorp.org/secure/Tests.jspa\#/testPlan/LVV-P72}{LVV-P72} in Jira is \textbf{ Draft }.

\subsection{References}
\label{sect:references}
\renewcommand{\refname}{}
\bibliography{lsst,refs,books,refs_ads,local}


\newpage
\section{Test Plan Details}
\label{sect:testplan}


\subsection{Data Collection}

  Observing is not required for this test campaign.

\subsection{Verification Environment}
\label{sect:hwconf}
  Tests that require code and/or data analysis will use the
``lsst-lsp-stable'' instance of the Rubin Observatory/LSST Science
Platform (LSP), hosted at the LDF, and the ``lsst-dev'' development
cluster at NCSA.




\subsection{Related Documentation}

The documentation related to this test campaign should be provided in the following DocuShare Collection
(as per Verification Artifacts in Jira test plan LVV-P72).

\begin{itemize}
\item DocuShare Collection Not Specified
\end{itemize}



\subsection{PMCS Activity}

Primavera milestones related to the test campaign.

\begin{itemize}
\item None
\end{itemize}


\newpage
\section{Personnel}
\label{sect:personnel}

The personnel involved in the test campaign is shown in the following table.

{\small
\begin{longtable}{p{3cm}p{3cm}p{3cm}p{6cm}}
\hline
\multicolumn{2}{r}{T. Plan \href{https://jira.lsstcorp.org/secure/Tests.jspa\#/testPlan/LVV-P72}{LVV-P72} owner:} &
\multicolumn{2}{l}{\textbf{ Jeffrey Carlin } }\\\hline
\multicolumn{2}{r}{T. Cycle \href{https://jira.lsstcorp.org/secure/Tests.jspa\#/testCycle/LVV-C154}{LVV-C154} owner:} &
\multicolumn{2}{l}{\textbf{
Jeffrey Carlin }
} \\\hline
\textbf{Test Cases} & \textbf{Assigned to} & \textbf{Executed by} & \textbf{Additional Test Personnel} \\ \hline
\href{https://jira.lsstcorp.org/secure/Tests.jspa#/testCase/LVV-T190}{LVV-T190}
& {\small Robert Gruendl } & {\small  } &
\begin{minipage}[]{6cm}
\smallskip
{\small  }
\medskip
\end{minipage}
\\ \hline
\href{https://jira.lsstcorp.org/secure/Tests.jspa#/testCase/LVV-T191}{LVV-T191}
& {\small Robert Gruendl } & {\small  } &
\begin{minipage}[]{6cm}
\smallskip
{\small  }
\medskip
\end{minipage}
\\ \hline
\href{https://jira.lsstcorp.org/secure/Tests.jspa#/testCase/LVV-T1830}{LVV-T1830}
& {\small Jeffrey Carlin } & {\small  } &
\begin{minipage}[]{6cm}
\smallskip
{\small  }
\medskip
\end{minipage}
\\ \hline
\href{https://jira.lsstcorp.org/secure/Tests.jspa#/testCase/LVV-T29}{LVV-T29}
& {\small Kian-Tat Lim } & {\small  } &
\begin{minipage}[]{6cm}
\smallskip
{\small  }
\medskip
\end{minipage}
\\ \hline
\href{https://jira.lsstcorp.org/secure/Tests.jspa#/testCase/LVV-T32}{LVV-T32}
& {\small Kian-Tat Lim } & {\small  } &
\begin{minipage}[]{6cm}
\smallskip
{\small  }
\medskip
\end{minipage}
\\ \hline
\href{https://jira.lsstcorp.org/secure/Tests.jspa#/testCase/LVV-T84}{LVV-T84}
& {\small Robert Lupton } & {\small  } &
\begin{minipage}[]{6cm}
\smallskip
{\small  }
\medskip
\end{minipage}
\\ \hline
\href{https://jira.lsstcorp.org/secure/Tests.jspa#/testCase/LVV-T85}{LVV-T85}
& {\small Robert Lupton } & {\small  } &
\begin{minipage}[]{6cm}
\smallskip
{\small  }
\medskip
\end{minipage}
\\ \hline
\href{https://jira.lsstcorp.org/secure/Tests.jspa#/testCase/LVV-T88}{LVV-T88}
& {\small Robert Lupton } & {\small  } &
\begin{minipage}[]{6cm}
\smallskip
{\small  }
\medskip
\end{minipage}
\\ \hline
\href{https://jira.lsstcorp.org/secure/Tests.jspa#/testCase/LVV-T115}{LVV-T115}
& {\small Kian-Tat Lim } & {\small  } &
\begin{minipage}[]{6cm}
\smallskip
{\small  }
\medskip
\end{minipage}
\\ \hline
\end{longtable}
}

\newpage

\section{Test Campaign Overview}
\label{sect:overview}

\subsection{Summary}
\label{sect:summarytable}

{\small
\begin{longtable}{p{2cm}cp{2.3cm}p{8.6cm}p{2.3cm}}
\toprule
\multicolumn{2}{r}{ T. Plan \href{https://jira.lsstcorp.org/secure/Tests.jspa\#/testPlan/LVV-P72}{LVV-P72}:} &
\multicolumn{2}{p{10.9cm}}{\textbf{ DM Acceptance Testing, Operations Rehearsal \#2 }} & Draft \\\hline
\multicolumn{2}{r}{ T. Cycle \href{https://jira.lsstcorp.org/secure/Tests.jspa\#/testCycle/LVV-C154}{LVV-C154}:} &
\multicolumn{2}{p{10.9cm}}{\textbf{ DM Acceptance Testing, Operations Rehearsal \#2 }} & Not Executed \\\hline
\textbf{Test Cases} &  \textbf{Ver.} & \textbf{Status} & \textbf{Comment} & \textbf{Issues} \\\toprule
\href{https://jira.lsstcorp.org/secure/Tests.jspa#/testCase/LVV-T190}{LVV-T190}
&  1
& Not Executed &
\begin{minipage}[]{9cm}
\smallskip

\medskip
\end{minipage}
&
\\\hline
\href{https://jira.lsstcorp.org/secure/Tests.jspa#/testCase/LVV-T191}{LVV-T191}
&  1
& Not Executed &
\begin{minipage}[]{9cm}
\smallskip

\medskip
\end{minipage}
&
\\\hline
\href{https://jira.lsstcorp.org/secure/Tests.jspa#/testCase/LVV-T1830}{LVV-T1830}
&  1
& Not Executed &
\begin{minipage}[]{9cm}
\smallskip

\medskip
\end{minipage}
&
\\\hline
\href{https://jira.lsstcorp.org/secure/Tests.jspa#/testCase/LVV-T29}{LVV-T29}
&  1
& Not Executed &
\begin{minipage}[]{9cm}
\smallskip

\medskip
\end{minipage}
&
\\\hline
\href{https://jira.lsstcorp.org/secure/Tests.jspa#/testCase/LVV-T32}{LVV-T32}
&  1
& Not Executed &
\begin{minipage}[]{9cm}
\smallskip

\medskip
\end{minipage}
&
\\\hline
\href{https://jira.lsstcorp.org/secure/Tests.jspa#/testCase/LVV-T84}{LVV-T84}
&  1
& Not Executed &
\begin{minipage}[]{9cm}
\smallskip

\medskip
\end{minipage}
&
\\\hline
\href{https://jira.lsstcorp.org/secure/Tests.jspa#/testCase/LVV-T85}{LVV-T85}
&  1
& Not Executed &
\begin{minipage}[]{9cm}
\smallskip

\medskip
\end{minipage}
&
\\\hline
\href{https://jira.lsstcorp.org/secure/Tests.jspa#/testCase/LVV-T88}{LVV-T88}
&  1
& Not Executed &
\begin{minipage}[]{9cm}
\smallskip

\medskip
\end{minipage}
&
\\\hline
\href{https://jira.lsstcorp.org/secure/Tests.jspa#/testCase/LVV-T115}{LVV-T115}
&  1
& Not Executed &
\begin{minipage}[]{9cm}
\smallskip

\medskip
\end{minipage}
&
\\\hline
\caption{Test Campaign Summary}
\label{table:summary}
\end{longtable}
}

\subsection{Overall Assessment}
\label{sect:overallassessment}

Not yet available.

\subsection{Recommended Improvements}
\label{sect:recommendations}

Not yet available.

\newpage
\section{Detailed Test Results}
\label{sect:detailedtestresults}

\subsection{Test Cycle LVV-C154 }

Open test cycle {\it \href{https://jira.lsstcorp.org/secure/Tests.jspa#/testrun/LVV-C154}{DM Acceptance Testing, Operations Rehearsal \#2}} in Jira.

Test Cycle name: DM Acceptance Testing, Operations Rehearsal \#2\\
Status: Not Executed

This test cycle verifies a subset of
\href{https://lse-61.lsst.io/}{DMSR} (\citeds{LSE-61}) requirements in order to
verify their completion and readiness for LSST Operations (i.e., that
the requirements laid out in \citeds{LSE-61} have been met by the DM Systems).
These acceptance tests are to be carried out during DM Operations
Rehearsal \#2.

\subsubsection{Software Version/Baseline}
Not provided.

\subsubsection{Configuration}
Not provided.

\subsubsection{Test Cases in LVV-C154 Test Cycle}

\paragraph{ LVV-T190 - Verify implementation of Base Facility Co-Location with Existing
Facility }\mbox{}\\

Version \textbf{1}.
Open  \href{https://jira.lsstcorp.org/secure/Tests.jspa#/testCase/LVV-T190}{\textit{ LVV-T190 } }
test case in Jira.

Verify that the Base Facility is located at an existing known supported
facility.

\textbf{ Preconditions}:\\


Execution status: {\bf Not Executed }

Final comment:\\


Detailed steps results:

\begin{longtable}{p{1cm}p{15cm}}
\hline
{Step} & Step Details\\ \hline
1 & Description \\
 & \begin{minipage}[t]{15cm}
{\footnotesize
Analyze design

\medskip }
\end{minipage}
\\ \cdashline{2-2}


 & Expected Result \\
 & \begin{minipage}[t]{15cm}{\footnotesize

\medskip }
\end{minipage} \\ \cdashline{2-2}

 & Actual Result \\
 & \begin{minipage}[t]{15cm}{\footnotesize

\medskip }
\end{minipage} \\ \cdashline{2-2}

 & Status: \textbf{ Not Executed } \\ \hline

\end{longtable}

\paragraph{ LVV-T191 - Verify implementation of Commissioning Cluster }\mbox{}\\

Version \textbf{1}.
Open  \href{https://jira.lsstcorp.org/secure/Tests.jspa#/testCase/LVV-T191}{\textit{ LVV-T191 } }
test case in Jira.

Verify that the Commissioning Cluster has sufficient Compute/Storage/LAN
at the Base Facility to support Commissioning.

\textbf{ Preconditions}:\\


Execution status: {\bf Not Executed }

Final comment:\\


Detailed steps results:

\begin{longtable}{p{1cm}p{15cm}}
\hline
{Step} & Step Details\\ \hline
1 & Description \\
 & \begin{minipage}[t]{15cm}
{\footnotesize
Analyze design and budget

\medskip }
\end{minipage}
\\ \cdashline{2-2}


 & Expected Result \\
 & \begin{minipage}[t]{15cm}{\footnotesize

\medskip }
\end{minipage} \\ \cdashline{2-2}

 & Actual Result \\
 & \begin{minipage}[t]{15cm}{\footnotesize

\medskip }
\end{minipage} \\ \cdashline{2-2}

 & Status: \textbf{ Not Executed } \\ \hline

\end{longtable}

\paragraph{ LVV-T1830 - Verify Implementation of Scientific Visualization of Camera Image Data }\mbox{}\\

Version \textbf{1}.
Open  \href{https://jira.lsstcorp.org/secure/Tests.jspa#/testCase/LVV-T1830}{\textit{ LVV-T1830 } }
test case in Jira.

Verify that all scientific visualization of camera image data uses the
coordinate systems defined in \href{https://lse-349.lsst.io/}{LSE-349}.

\textbf{ Preconditions}:\\


Execution status: {\bf Not Executed }

Final comment:\\


Detailed steps results:

\begin{longtable}{p{1cm}p{15cm}}
\hline
{Step} & Step Details\\ \hline
1 & Description \\
 & \begin{minipage}[t]{15cm}
{\footnotesize

\medskip }
\end{minipage}
\\ \cdashline{2-2}


 & Expected Result \\
 & \begin{minipage}[t]{15cm}{\footnotesize

\medskip }
\end{minipage} \\ \cdashline{2-2}

 & Actual Result \\
 & \begin{minipage}[t]{15cm}{\footnotesize

\medskip }
\end{minipage} \\ \cdashline{2-2}

 & Status: \textbf{ Not Executed } \\ \hline

\end{longtable}

\paragraph{ LVV-T29 - Verify implementation of Raw Science Image Data Acquisition }\mbox{}\\

Version \textbf{1}.
Open  \href{https://jira.lsstcorp.org/secure/Tests.jspa#/testCase/LVV-T29}{\textit{ LVV-T29 } }
test case in Jira.

Verify acquisition of raw data from L1 Test Stand DAQ while simulating
all modes

\textbf{ Preconditions}:\\


Execution status: {\bf Not Executed }

Final comment:\\


Detailed steps results:

\begin{longtable}{p{1cm}p{15cm}}
\hline
{Step} & Step Details\\ \hline
1 & Description \\
 & \begin{minipage}[t]{15cm}
{\footnotesize
{Ingest raw data from L1 Test Stand DAQ, simulating each observing
mode\\
}

\medskip }
\end{minipage}
\\ \cdashline{2-2}


 & Expected Result \\
 & \begin{minipage}[t]{15cm}{\footnotesize

\medskip }
\end{minipage} \\ \cdashline{2-2}

 & Actual Result \\
 & \begin{minipage}[t]{15cm}{\footnotesize

\medskip }
\end{minipage} \\ \cdashline{2-2}

 & Status: \textbf{ Not Executed } \\ \hline

2 & Description \\
 & \begin{minipage}[t]{15cm}
{\footnotesize
O{bserve image and its metadata is present and queryable in the Data
Backbone.}

\medskip }
\end{minipage}
\\ \cdashline{2-2}


 & Expected Result \\
 & \begin{minipage}[t]{15cm}{\footnotesize
Well-formed image data with appropriate associated metadata.

\medskip }
\end{minipage} \\ \cdashline{2-2}

 & Actual Result \\
 & \begin{minipage}[t]{15cm}{\footnotesize

\medskip }
\end{minipage} \\ \cdashline{2-2}

 & Status: \textbf{ Not Executed } \\ \hline

\end{longtable}

\paragraph{ LVV-T32 - Verify implementation of Raw Image Assembly }\mbox{}\\

Version \textbf{1}.
Open  \href{https://jira.lsstcorp.org/secure/Tests.jspa#/testCase/LVV-T32}{\textit{ LVV-T32 } }
test case in Jira.

Verify that the raw exposure data from all readout channels in a sensor
can be assembled into a single image, and that all required/relevant
metadata are associated with the image data.~

\textbf{ Preconditions}:\\


Execution status: {\bf Not Executed }

Final comment:\\


Detailed steps results:

\begin{longtable}{p{1cm}p{15cm}}
\hline
{Step} & Step Details\\ \hline
1 & Description \\
 & \begin{minipage}[t]{15cm}
{\footnotesize
Ingest data from the L1 Camera Test Stand DAQ.

\medskip }
\end{minipage}
\\ \cdashline{2-2}


 & Expected Result \\
 & \begin{minipage}[t]{15cm}{\footnotesize

\medskip }
\end{minipage} \\ \cdashline{2-2}

 & Actual Result \\
 & \begin{minipage}[t]{15cm}{\footnotesize

\medskip }
\end{minipage} \\ \cdashline{2-2}

 & Status: \textbf{ Not Executed } \\ \hline

2 & Description \\
 & \begin{minipage}[t]{15cm}
{\footnotesize
Simulate all different modes of data gathering.

\medskip }
\end{minipage}
\\ \cdashline{2-2}


 & Expected Result \\
 & \begin{minipage}[t]{15cm}{\footnotesize

\medskip }
\end{minipage} \\ \cdashline{2-2}

 & Actual Result \\
 & \begin{minipage}[t]{15cm}{\footnotesize

\medskip }
\end{minipage} \\ \cdashline{2-2}

 & Status: \textbf{ Not Executed } \\ \hline

3 & Description \\
 & \begin{minipage}[t]{15cm}
{\footnotesize
Verify that a raw image is constructed in correct format.

\medskip }
\end{minipage}
\\ \cdashline{2-2}


 & Expected Result \\
 & \begin{minipage}[t]{15cm}{\footnotesize
A single raw image combining data from all readout channels for a given
sensor.~

\medskip }
\end{minipage} \\ \cdashline{2-2}

 & Actual Result \\
 & \begin{minipage}[t]{15cm}{\footnotesize

\medskip }
\end{minipage} \\ \cdashline{2-2}

 & Status: \textbf{ Not Executed } \\ \hline

4 & Description \\
 & \begin{minipage}[t]{15cm}
{\footnotesize
Verify that a raw image is constructed with correct metadata.

\medskip }
\end{minipage}
\\ \cdashline{2-2}


 & Expected Result \\
 & \begin{minipage}[t]{15cm}{\footnotesize
Image header or ancillary table contains the required metadata about the
observing context in which data were gathered.

\medskip }
\end{minipage} \\ \cdashline{2-2}

 & Actual Result \\
 & \begin{minipage}[t]{15cm}{\footnotesize

\medskip }
\end{minipage} \\ \cdashline{2-2}

 & Status: \textbf{ Not Executed } \\ \hline

\end{longtable}

\paragraph{ LVV-T84 - Verify implementation of Bias Residual Image }\mbox{}\\

Version \textbf{1}.
Open  \href{https://jira.lsstcorp.org/secure/Tests.jspa#/testCase/LVV-T84}{\textit{ LVV-T84 } }
test case in Jira.

Verify that DMS can construct a bias residual image that corrects for
temporally-stable bias structures.\\
Verify that DMS can do this on demand.

\textbf{ Preconditions}:\\


Execution status: {\bf Not Executed }

Final comment:\\


Detailed steps results:

\begin{longtable}{p{1cm}p{15cm}}
\hline
{Step} & Step Details\\ \hline
1 & Description \\
 & \begin{minipage}[t]{15cm}
{\footnotesize
Identify the location of an appropriate precursor dataset.

\medskip }
\end{minipage}
\\ \cdashline{2-2}


 & Expected Result \\
 & \begin{minipage}[t]{15cm}{\footnotesize

\medskip }
\end{minipage} \\ \cdashline{2-2}

 & Actual Result \\
 & \begin{minipage}[t]{15cm}{\footnotesize

\medskip }
\end{minipage} \\ \cdashline{2-2}

 & Status: \textbf{ Not Executed } \\ \hline

2 & Description \\
 & \begin{minipage}[t]{15cm}
{\footnotesize
Identify the path to the data repository, which we will refer to as
`DATA/path', then execute the following:

\medskip }
\end{minipage}
\\ \cdashline{2-2}

 & Example Code \\
 & \begin{minipage}[t]{15cm}{\footnotesize
\begin{verbatim}
import lsst.daf.persistence as dafPersist
butler = dafPersist.Butler(inputs='DATA/path')
\end{verbatim}

\medskip }
\end{minipage} \\ \cdashline{2-2}

 & Expected Result \\
 & \begin{minipage}[t]{15cm}{\footnotesize
Butler repo available for reading.

\medskip }
\end{minipage} \\ \cdashline{2-2}

 & Actual Result \\
 & \begin{minipage}[t]{15cm}{\footnotesize

\medskip }
\end{minipage} \\ \cdashline{2-2}

 & Status: \textbf{ Not Executed } \\ \hline

3 & Description \\
 & \begin{minipage}[t]{15cm}
{\footnotesize
Import the standard libraries required for the rest of this test:

\medskip }
\end{minipage}
\\ \cdashline{2-2}

 & Example Code \\
 & \begin{minipage}[t]{15cm}{\footnotesize
import osimport lsst.afw.display as afwDisplay\\
from lsst.daf.persistence import Butler\\
from lsst.ip.isr import IsrTask\\
from firefly\_client import FireflyClient\\
from IPython.display import IFrame

\medskip }
\end{minipage} \\ \cdashline{2-2}

 & Expected Result \\
 & \begin{minipage}[t]{15cm}{\footnotesize

\medskip }
\end{minipage} \\ \cdashline{2-2}

 & Actual Result \\
 & \begin{minipage}[t]{15cm}{\footnotesize

\medskip }
\end{minipage} \\ \cdashline{2-2}

 & Status: \textbf{ Not Executed } \\ \hline

4 & Description \\
 & \begin{minipage}[t]{15cm}
{\footnotesize
Ingest the dataset from step 1 using the Butler (e.g., following example
code below).

\medskip }
\end{minipage}
\\ \cdashline{2-2}

 & Example Code \\
 & \begin{minipage}[t]{15cm}{\footnotesize
butler = Butler(\$REPOSITORY\_PATH)\\
raw = butler.get(``raw'', visit=\$VISIT\_ID, detector=2)\\
bias = butler.get(``bias'', visit=\$VISIT\_ID, detector=2)

\medskip }
\end{minipage} \\ \cdashline{2-2}

 & Expected Result \\
 & \begin{minipage}[t]{15cm}{\footnotesize

\medskip }
\end{minipage} \\ \cdashline{2-2}

 & Actual Result \\
 & \begin{minipage}[t]{15cm}{\footnotesize

\medskip }
\end{minipage} \\ \cdashline{2-2}

 & Status: \textbf{ Not Executed } \\ \hline

5 & Description \\
 & \begin{minipage}[t]{15cm}
{\footnotesize
Display the bias image and inspect that its pixels contain unique
values.

\medskip }
\end{minipage}
\\ \cdashline{2-2}


 & Expected Result \\
 & \begin{minipage}[t]{15cm}{\footnotesize
A relatively flat image showing the bias level with roughly Poisson
noise.

\medskip }
\end{minipage} \\ \cdashline{2-2}

 & Actual Result \\
 & \begin{minipage}[t]{15cm}{\footnotesize

\medskip }
\end{minipage} \\ \cdashline{2-2}

 & Status: \textbf{ Not Executed } \\ \hline

6 & Description \\
 & \begin{minipage}[t]{15cm}
{\footnotesize
Configure and run an Instrument Signature Removal (ISR) task on the raw
data. Most corrections are disabled for simplicity, but the bias frame
is applied.\\[2\baselineskip]

\medskip }
\end{minipage}
\\ \cdashline{2-2}

 & Example Code \\
 & \begin{minipage}[t]{15cm}{\footnotesize
isr\_config = IsrTask.ConfigClass()\\
isr\_config.doDark=False\\
isr\_config.doFlat=False\\
isr\_config.doFringe=False\\
isr\_config.doDefect=False\\
isr\_config.doAddDistortionModel=False\\
isr\_config.doLinearize=False\\
isr = IsrTask(config=isr\_config)\\
result = isr.run(raw, bias=bias)

\medskip }
\end{minipage} \\ \cdashline{2-2}

 & Expected Result \\
 & \begin{minipage}[t]{15cm}{\footnotesize
A trimmed, bias-corrected image in `result`.

\medskip }
\end{minipage} \\ \cdashline{2-2}

 & Actual Result \\
 & \begin{minipage}[t]{15cm}{\footnotesize

\medskip }
\end{minipage} \\ \cdashline{2-2}

 & Status: \textbf{ Not Executed } \\ \hline

7 & Description \\
 & \begin{minipage}[t]{15cm}
{\footnotesize
Display the `result` image and confirm that the bias correction has been
performed.

\medskip }
\end{minipage}
\\ \cdashline{2-2}


 & Expected Result \\
 & \begin{minipage}[t]{15cm}{\footnotesize
A displayed image with bias removed (i.e., typical background counts
reduced relative to the raw frame).

\medskip }
\end{minipage} \\ \cdashline{2-2}

 & Actual Result \\
 & \begin{minipage}[t]{15cm}{\footnotesize

\medskip }
\end{minipage} \\ \cdashline{2-2}

 & Status: \textbf{ Not Executed } \\ \hline

\end{longtable}

\paragraph{ LVV-T85 - Verify implementation of Crosstalk Correction Matrix }\mbox{}\\

Version \textbf{1}.
Open  \href{https://jira.lsstcorp.org/secure/Tests.jspa#/testCase/LVV-T85}{\textit{ LVV-T85 } }
test case in Jira.

Verify that the DMS can generate a cross-talk correction matrix from
appropriate calibration data.\\
Verify that the DMS can measure the effectiveness of the cross-talk
correction matrix.

\textbf{ Preconditions}:\\


Execution status: {\bf Not Executed }

Final comment:\\


Detailed steps results:

\begin{longtable}{p{1cm}p{15cm}}
\hline
{Step} & Step Details\\ \hline
1 & Description \\
 & \begin{minipage}[t]{15cm}
{\footnotesize
Identify an appropriate calibration dataset that can be used to derive
the crosstalk correction matrix.

\medskip }
\end{minipage}
\\ \cdashline{2-2}


 & Expected Result \\
 & \begin{minipage}[t]{15cm}{\footnotesize

\medskip }
\end{minipage} \\ \cdashline{2-2}

 & Actual Result \\
 & \begin{minipage}[t]{15cm}{\footnotesize

\medskip }
\end{minipage} \\ \cdashline{2-2}

 & Status: \textbf{ Not Executed } \\ \hline

2 & Description \\
 & \begin{minipage}[t]{15cm}
{\footnotesize
Execute the Calibration Products Production payload. The payload uses
raw calibration images and information from the Transformed EFD to
generate a subset of Master Calibration Images and Calibration Database
entries in the Data Backbone.

\medskip }
\end{minipage}
\\ \cdashline{2-2}


 & Expected Result \\
 & \begin{minipage}[t]{15cm}{\footnotesize

\medskip }
\end{minipage} \\ \cdashline{2-2}

 & Actual Result \\
 & \begin{minipage}[t]{15cm}{\footnotesize

\medskip }
\end{minipage} \\ \cdashline{2-2}

 & Status: \textbf{ Not Executed } \\ \hline

3 & Description \\
 & \begin{minipage}[t]{15cm}
{\footnotesize
Confirm that the expected Master Calibration images and Calibration
Database entries are present and well-formed.

\medskip }
\end{minipage}
\\ \cdashline{2-2}


 & Expected Result \\
 & \begin{minipage}[t]{15cm}{\footnotesize

\medskip }
\end{minipage} \\ \cdashline{2-2}

 & Actual Result \\
 & \begin{minipage}[t]{15cm}{\footnotesize

\medskip }
\end{minipage} \\ \cdashline{2-2}

 & Status: \textbf{ Not Executed } \\ \hline

4 & Description \\
 & \begin{minipage}[t]{15cm}
{\footnotesize
Confirm that the crosstalk correction matrix is produced and persisted.

\medskip }
\end{minipage}
\\ \cdashline{2-2}


 & Expected Result \\
 & \begin{minipage}[t]{15cm}{\footnotesize
A correction matrix quantifying what fraction of the signal detected in
any given amplifier on each sensor in the focal plane appears in any
other amplifier.

\medskip }
\end{minipage} \\ \cdashline{2-2}

 & Actual Result \\
 & \begin{minipage}[t]{15cm}{\footnotesize

\medskip }
\end{minipage} \\ \cdashline{2-2}

 & Status: \textbf{ Not Executed } \\ \hline

5 & Description \\
 & \begin{minipage}[t]{15cm}
{\footnotesize
Apply the crosstalk correction to simulated images, and confirm that the
correction is performing as expected.

\medskip }
\end{minipage}
\\ \cdashline{2-2}


 & Expected Result \\
 & \begin{minipage}[t]{15cm}{\footnotesize
A noticeable difference between images before and after applying the
correction.

\medskip }
\end{minipage} \\ \cdashline{2-2}

 & Actual Result \\
 & \begin{minipage}[t]{15cm}{\footnotesize

\medskip }
\end{minipage} \\ \cdashline{2-2}

 & Status: \textbf{ Not Executed } \\ \hline

\end{longtable}

\paragraph{ LVV-T88 - Verify implementation of Calibration Data Products }\mbox{}\\

Version \textbf{1}.
Open  \href{https://jira.lsstcorp.org/secure/Tests.jspa#/testCase/LVV-T88}{\textit{ LVV-T88 } }
test case in Jira.

Verify that the DMS can produce and archive the required Calibration
Data Products: cross talk correction, bias, dark, monochromatic dome
flats, broad-band flats, fringe correction, and illumination
corrections.

\textbf{ Preconditions}:\\


Execution status: {\bf Not Executed }

Final comment:\\


Detailed steps results:

\begin{longtable}{p{1cm}p{15cm}}
\hline
{Step} & Step Details\\ \hline
1 & Description \\
 & \begin{minipage}[t]{15cm}
{\footnotesize
Identify a suitable set of calibration frames, including biases, dark
frames, and flat-field frames.

\medskip }
\end{minipage}
\\ \cdashline{2-2}


 & Expected Result \\
 & \begin{minipage}[t]{15cm}{\footnotesize

\medskip }
\end{minipage} \\ \cdashline{2-2}

 & Actual Result \\
 & \begin{minipage}[t]{15cm}{\footnotesize

\medskip }
\end{minipage} \\ \cdashline{2-2}

 & Status: \textbf{ Not Executed } \\ \hline

2 & Description \\
 & \begin{minipage}[t]{15cm}
{\footnotesize
Execute the Calibration Products Production payload. The payload uses
raw calibration images and information from the Transformed EFD to
generate a subset of Master Calibration Images and Calibration Database
entries in the Data Backbone.

\medskip }
\end{minipage}
\\ \cdashline{2-2}


 & Expected Result \\
 & \begin{minipage}[t]{15cm}{\footnotesize

\medskip }
\end{minipage} \\ \cdashline{2-2}

 & Actual Result \\
 & \begin{minipage}[t]{15cm}{\footnotesize

\medskip }
\end{minipage} \\ \cdashline{2-2}

 & Status: \textbf{ Not Executed } \\ \hline

3 & Description \\
 & \begin{minipage}[t]{15cm}
{\footnotesize
Confirm that the expected Master Calibration images and Calibration
Database entries are present and well-formed.

\medskip }
\end{minipage}
\\ \cdashline{2-2}


 & Expected Result \\
 & \begin{minipage}[t]{15cm}{\footnotesize

\medskip }
\end{minipage} \\ \cdashline{2-2}

 & Actual Result \\
 & \begin{minipage}[t]{15cm}{\footnotesize

\medskip }
\end{minipage} \\ \cdashline{2-2}

 & Status: \textbf{ Not Executed } \\ \hline

4 & Description \\
 & \begin{minipage}[t]{15cm}
{\footnotesize
Confirm that the expected data products are created, and that they have
the expected properties.

\medskip }
\end{minipage}
\\ \cdashline{2-2}


 & Expected Result \\
 & \begin{minipage}[t]{15cm}{\footnotesize
A full set of calibration data products has been created, and they are
well-formed.

\medskip }
\end{minipage} \\ \cdashline{2-2}

 & Actual Result \\
 & \begin{minipage}[t]{15cm}{\footnotesize

\medskip }
\end{minipage} \\ \cdashline{2-2}

 & Status: \textbf{ Not Executed } \\ \hline

5 & Description \\
 & \begin{minipage}[t]{15cm}
{\footnotesize
Test that the calibration products are archived, and can readily be
applied to science data to produce the desired corrections.

\medskip }
\end{minipage}
\\ \cdashline{2-2}


 & Expected Result \\
 & \begin{minipage}[t]{15cm}{\footnotesize
Confirmation that application of the calibration products to processed
data has the desired effects.

\medskip }
\end{minipage} \\ \cdashline{2-2}

 & Actual Result \\
 & \begin{minipage}[t]{15cm}{\footnotesize

\medskip }
\end{minipage} \\ \cdashline{2-2}

 & Status: \textbf{ Not Executed } \\ \hline

\end{longtable}

\paragraph{ LVV-T115 - Verify implementation of Calibration Production Processing }\mbox{}\\

Version \textbf{1}.
Open  \href{https://jira.lsstcorp.org/secure/Tests.jspa#/testCase/LVV-T115}{\textit{ LVV-T115 } }
test case in Jira.

Execute CPP on a variety of representative cadences, and verify that the
calibration pipeline correctly produces necessary calibration products.

\textbf{ Preconditions}:\\


Execution status: {\bf Not Executed }

Final comment:\\


Detailed steps results:

\begin{longtable}{p{1cm}p{15cm}}
\hline
{Step} & Step Details\\ \hline
1 & Description \\
 & \begin{minipage}[t]{15cm}
{\footnotesize
Identify a suitable set of calibration frames, including biases, dark
frames, and flat-field frames.

\medskip }
\end{minipage}
\\ \cdashline{2-2}


 & Expected Result \\
 & \begin{minipage}[t]{15cm}{\footnotesize

\medskip }
\end{minipage} \\ \cdashline{2-2}

 & Actual Result \\
 & \begin{minipage}[t]{15cm}{\footnotesize

\medskip }
\end{minipage} \\ \cdashline{2-2}

 & Status: \textbf{ Not Executed } \\ \hline

2 & Description \\
 & \begin{minipage}[t]{15cm}
{\footnotesize
Execute the Calibration Products Production payload. The payload uses
raw calibration images and information from the Transformed EFD to
generate a subset of Master Calibration Images and Calibration Database
entries in the Data Backbone.

\medskip }
\end{minipage}
\\ \cdashline{2-2}


 & Expected Result \\
 & \begin{minipage}[t]{15cm}{\footnotesize

\medskip }
\end{minipage} \\ \cdashline{2-2}

 & Actual Result \\
 & \begin{minipage}[t]{15cm}{\footnotesize

\medskip }
\end{minipage} \\ \cdashline{2-2}

 & Status: \textbf{ Not Executed } \\ \hline

3 & Description \\
 & \begin{minipage}[t]{15cm}
{\footnotesize
Confirm that the expected Master Calibration images and Calibration
Database entries are present and well-formed.

\medskip }
\end{minipage}
\\ \cdashline{2-2}


 & Expected Result \\
 & \begin{minipage}[t]{15cm}{\footnotesize

\medskip }
\end{minipage} \\ \cdashline{2-2}

 & Actual Result \\
 & \begin{minipage}[t]{15cm}{\footnotesize

\medskip }
\end{minipage} \\ \cdashline{2-2}

 & Status: \textbf{ Not Executed } \\ \hline

4 & Description \\
 & \begin{minipage}[t]{15cm}
{\footnotesize
Confirm that the expected data products are created, and that they have
the expected properties.

\medskip }
\end{minipage}
\\ \cdashline{2-2}


 & Expected Result \\
 & \begin{minipage}[t]{15cm}{\footnotesize
Repos containing valid calibration products that are well-formed and
ready to be applied to processed datasets.

\medskip }
\end{minipage} \\ \cdashline{2-2}

 & Actual Result \\
 & \begin{minipage}[t]{15cm}{\footnotesize

\medskip }
\end{minipage} \\ \cdashline{2-2}

 & Status: \textbf{ Not Executed } \\ \hline

\end{longtable}


\newpage
\appendix
%Make sure lsst-texmf/bin/generateAcronyms.py is in your path
\section{Acronyms used in this document}\label{sec:acronyms}
\input{acronyms.tex}

\newpage
% generated from JIRA project LVV
% using template at <template>.
% using docsteady version 1.2rc24.post10+g05696ce.d20201201
% Please do not edit -- update information in Jira instead

\section{Traceability}

\begin{longtable}{p{3cm}p{3cm}L{9cm}}
\hline
\textbf{Test Case} & \textbf{VE Key} & \textbf{VE Summary} \\ \hline
\href{https://jira.lsstcorp.org/secure/Tests.jspa#/testCase/LVV-T1934}{LVV-T1934} &
  \href{https://jira.lsstcorp.org/browse/LVV-8}{LVV-8}
  & DMS-REQ-0018-V-01: Raw Science Image Data Acquisition
 \\ \cdashline{2-3}
 &   \href{https://jira.lsstcorp.org/browse/LVV-11}{LVV-11}
  & DMS-REQ-0024-V-01: Raw Image Assembly
 \\ \cdashline{2-3}
 &   \href{https://jira.lsstcorp.org/browse/LVV-177}{LVV-177}
  & DMS-REQ-0346-V-01: Data Availability
 \\ \cdashline{2-3}
 &   \href{https://jira.lsstcorp.org/browse/LVV-130}{LVV-130}
  & DMS-REQ-0299-V-01: Data Product Ingest
 \\ \cdashline{2-3}
\hline
\href{https://jira.lsstcorp.org/secure/Tests.jspa#/testCase/LVV-T1935}{LVV-T1935} &
  \href{https://jira.lsstcorp.org/browse/LVV-130}{LVV-130}
  & DMS-REQ-0299-V-01: Data Product Ingest
 \\ \cdashline{2-3}
 &   \href{https://jira.lsstcorp.org/browse/LVV-120}{LVV-120}
  & DMS-REQ-0289-V-01: Calibration Production Processing
 \\ \cdashline{2-3}
\hline
\end{longtable}


\end{document}
